\documentclass{article}[12pt]
\usepackage{color}
\usepackage[normalem]{ulem}
\usepackage{times}
\usepackage{fullpage}
\usepackage{amsmath}
\usepackage{amssymb}
\usepackage{tikz}
\def \R {\mathbb R}
\def \imp {\Longrightarrow}
\def \eps {\varepsilon}
\def \Inf {{\sf Inf}}
\newenvironment{proof}{{\bf Proof.  }}{\hfill$\Box$}
\newtheorem{theorem}{Theorem}[section]
\newtheorem{definition}{Definition}[section]
\newtheorem{corollary}{Corollary}[section]
\newtheorem{lemma}{Lemma}[section]
\newtheorem{claim}{Claim}[section]
\setlength {\parskip}{2pt}
\setlength{\parindent}{0pt}

\newcommand{\headings}[4]{\noindent {\bf Assignment 8 CME241} \hfill {{\bf Author:} Nicolas Sanchez} \\
{} \hfill {{\bf Due Date:} #2} \\

\rule[0.1in]{\textwidth}{0.025in}
}

\newcommand{\klnote}[1]{{\color{red} #1}}
\newcommand{\klsout}[1]{{\color{red} \sout{#1}}}

\begin{document}

\headings{\#1}{Tuesday, October 8, 10:30am}\section{} 



\section{Bank Lending MDP}
We define the following state space:
$$\mathbf{S} = \{ (c,w_d,t) | c, w_d \in \mathbb{R}, t\in \mathbb{N}, t \leq T\} $$
where $c$ is the cash at disposable of the bank, $w_d$ are deferred withdrawals that are backlogged from previous incapacity to pay and $t$ is the day number.\\

We denote $k(c) = K\text{cot}\frac{\pi c}{2C}$ the penalty for $c < C$ to simplify notation and have $r\in\mathbb{R}, r>0]$ a random variable denoting the return on the investment (so 0 is the investment becomes worthless) and $d\in\mathbb{R}$ is the random variable denoting the net deposit from clients (negative $d$ denotes net withdrawals). Then for any given state $s = (c,w_d,r)$ we have the action set:
$$ \mathbf{A} = \{ (b,i) | i \in [0,c - \tilde{k}(c)]\, b\in [0, \frac{(c-i-\tilde{k}(c))}{R}\}$$
where $b$ is the amount of cash borrowed by the bank and $i$ is the amount of cash invested in the risky asset. The transition is then defined by the probability distribution of the tuple $(c', w_d',t+1)$ where $c', w_d'$ are random variables defined as follows:
\begin{align*}
c' = \begin{cases} c-bR+ir -w_d + d &\text{ if $c-bR+ir-w_d \geq C$}\\ \max\{0, c-bR+ir+d-w_d -k(c-bR+ir+d)\} & \text{ otherwise}\end{cases}\\
w_d' = \begin{cases} 0 &\text{ if $c-bR+ir-w_d \geq C$} \\ - \min\{0, c-bR+ir+d-w_d -k(c-bR+ir+d)\}& \text{ otherwise} \end{cases}\\
\end{align*}
Finally the reward function is pretty straight forward looking only at the net Utility of assets minus liabilities at the end of the horizon:
$$ \mathbf{R}( (c,w_d, t)) =  \begin{cases} c-w_d &\text{ if $t=T$}\\ 0 & \text{ otherwise}\end{cases}$$

Since this clearly involves some rather intricate (stepwise functions) compositions of random variables and functions, this will likely require approximate dynamic programming to solve.

\section{Milk Vendor MDP}
We compute:
\begin{align*}
g(S) &= p\cdot g_1(S) + h g_2(S)\\
&= p\cdot \int_{S}^{\infty} (x-S)f(x)dx+ h \int_{-\infty}^S (S-x)f(x)dx\\
&= p \int_{S}^{\infty}xf(x)dx-pS\int_{S}^{\infty}f(x)dx- h \int_{-\infty}^Sxf(x)dx + hS\int_{-\infty}^Sf(x)dx\\
\end{align*}
Taking the derivative hence yields:

\begin{align*}
g'(S) &= -pSf(S)-p\int_{S}^{\infty}f(x)dx + pSf(S) - h Sf(S) + hSf(S)+h\int_{-\infty}^Sf(x)dx\\
&= -p(1-F(S)) +hF(S)\\
\end{align*}
and setting it to zero to find an extremum gives:
$$ F(S) = \frac{p}{h+p}$$
So the solution is the $S$ that satisfies the above (this exists since the the cumulative function $F$ is monotone increasing and is surjective onto $[0,1]$.\\

Looking at the payout structure suggests we can think of the milkman as having to choose a strike from which he will get long $p$ units of calls and $h$ units of puts where the underlying the demand for milk.

\section{American Option Pricing}
\end{document}
